% Options for packages loaded elsewhere
\PassOptionsToPackage{unicode}{hyperref}
\PassOptionsToPackage{hyphens}{url}
%
\documentclass[
]{article}
\usepackage{amsmath,amssymb}
\usepackage{lmodern}
\usepackage{iftex}
\ifPDFTeX
  \usepackage[T1]{fontenc}
  \usepackage[utf8]{inputenc}
  \usepackage{textcomp} % provide euro and other symbols
\else % if luatex or xetex
  \usepackage{unicode-math}
  \defaultfontfeatures{Scale=MatchLowercase}
  \defaultfontfeatures[\rmfamily]{Ligatures=TeX,Scale=1}
\fi
% Use upquote if available, for straight quotes in verbatim environments
\IfFileExists{upquote.sty}{\usepackage{upquote}}{}
\IfFileExists{microtype.sty}{% use microtype if available
  \usepackage[]{microtype}
  \UseMicrotypeSet[protrusion]{basicmath} % disable protrusion for tt fonts
}{}
\makeatletter
\@ifundefined{KOMAClassName}{% if non-KOMA class
  \IfFileExists{parskip.sty}{%
    \usepackage{parskip}
  }{% else
    \setlength{\parindent}{0pt}
    \setlength{\parskip}{6pt plus 2pt minus 1pt}}
}{% if KOMA class
  \KOMAoptions{parskip=half}}
\makeatother
\usepackage{xcolor}
\IfFileExists{xurl.sty}{\usepackage{xurl}}{} % add URL line breaks if available
\IfFileExists{bookmark.sty}{\usepackage{bookmark}}{\usepackage{hyperref}}
\hypersetup{
  pdftitle={Change in Head by Aquifer System},
  pdfauthor={Annette Hilton},
  hidelinks,
  pdfcreator={LaTeX via pandoc}}
\urlstyle{same} % disable monospaced font for URLs
\usepackage[margin=1in]{geometry}
\usepackage{color}
\usepackage{fancyvrb}
\newcommand{\VerbBar}{|}
\newcommand{\VERB}{\Verb[commandchars=\\\{\}]}
\DefineVerbatimEnvironment{Highlighting}{Verbatim}{commandchars=\\\{\}}
% Add ',fontsize=\small' for more characters per line
\usepackage{framed}
\definecolor{shadecolor}{RGB}{248,248,248}
\newenvironment{Shaded}{\begin{snugshade}}{\end{snugshade}}
\newcommand{\AlertTok}[1]{\textcolor[rgb]{0.94,0.16,0.16}{#1}}
\newcommand{\AnnotationTok}[1]{\textcolor[rgb]{0.56,0.35,0.01}{\textbf{\textit{#1}}}}
\newcommand{\AttributeTok}[1]{\textcolor[rgb]{0.77,0.63,0.00}{#1}}
\newcommand{\BaseNTok}[1]{\textcolor[rgb]{0.00,0.00,0.81}{#1}}
\newcommand{\BuiltInTok}[1]{#1}
\newcommand{\CharTok}[1]{\textcolor[rgb]{0.31,0.60,0.02}{#1}}
\newcommand{\CommentTok}[1]{\textcolor[rgb]{0.56,0.35,0.01}{\textit{#1}}}
\newcommand{\CommentVarTok}[1]{\textcolor[rgb]{0.56,0.35,0.01}{\textbf{\textit{#1}}}}
\newcommand{\ConstantTok}[1]{\textcolor[rgb]{0.00,0.00,0.00}{#1}}
\newcommand{\ControlFlowTok}[1]{\textcolor[rgb]{0.13,0.29,0.53}{\textbf{#1}}}
\newcommand{\DataTypeTok}[1]{\textcolor[rgb]{0.13,0.29,0.53}{#1}}
\newcommand{\DecValTok}[1]{\textcolor[rgb]{0.00,0.00,0.81}{#1}}
\newcommand{\DocumentationTok}[1]{\textcolor[rgb]{0.56,0.35,0.01}{\textbf{\textit{#1}}}}
\newcommand{\ErrorTok}[1]{\textcolor[rgb]{0.64,0.00,0.00}{\textbf{#1}}}
\newcommand{\ExtensionTok}[1]{#1}
\newcommand{\FloatTok}[1]{\textcolor[rgb]{0.00,0.00,0.81}{#1}}
\newcommand{\FunctionTok}[1]{\textcolor[rgb]{0.00,0.00,0.00}{#1}}
\newcommand{\ImportTok}[1]{#1}
\newcommand{\InformationTok}[1]{\textcolor[rgb]{0.56,0.35,0.01}{\textbf{\textit{#1}}}}
\newcommand{\KeywordTok}[1]{\textcolor[rgb]{0.13,0.29,0.53}{\textbf{#1}}}
\newcommand{\NormalTok}[1]{#1}
\newcommand{\OperatorTok}[1]{\textcolor[rgb]{0.81,0.36,0.00}{\textbf{#1}}}
\newcommand{\OtherTok}[1]{\textcolor[rgb]{0.56,0.35,0.01}{#1}}
\newcommand{\PreprocessorTok}[1]{\textcolor[rgb]{0.56,0.35,0.01}{\textit{#1}}}
\newcommand{\RegionMarkerTok}[1]{#1}
\newcommand{\SpecialCharTok}[1]{\textcolor[rgb]{0.00,0.00,0.00}{#1}}
\newcommand{\SpecialStringTok}[1]{\textcolor[rgb]{0.31,0.60,0.02}{#1}}
\newcommand{\StringTok}[1]{\textcolor[rgb]{0.31,0.60,0.02}{#1}}
\newcommand{\VariableTok}[1]{\textcolor[rgb]{0.00,0.00,0.00}{#1}}
\newcommand{\VerbatimStringTok}[1]{\textcolor[rgb]{0.31,0.60,0.02}{#1}}
\newcommand{\WarningTok}[1]{\textcolor[rgb]{0.56,0.35,0.01}{\textbf{\textit{#1}}}}
\usepackage{longtable,booktabs,array}
\usepackage{calc} % for calculating minipage widths
% Correct order of tables after \paragraph or \subparagraph
\usepackage{etoolbox}
\makeatletter
\patchcmd\longtable{\par}{\if@noskipsec\mbox{}\fi\par}{}{}
\makeatother
% Allow footnotes in longtable head/foot
\IfFileExists{footnotehyper.sty}{\usepackage{footnotehyper}}{\usepackage{footnote}}
\makesavenoteenv{longtable}
\usepackage{graphicx}
\makeatletter
\def\maxwidth{\ifdim\Gin@nat@width>\linewidth\linewidth\else\Gin@nat@width\fi}
\def\maxheight{\ifdim\Gin@nat@height>\textheight\textheight\else\Gin@nat@height\fi}
\makeatother
% Scale images if necessary, so that they will not overflow the page
% margins by default, and it is still possible to overwrite the defaults
% using explicit options in \includegraphics[width, height, ...]{}
\setkeys{Gin}{width=\maxwidth,height=\maxheight,keepaspectratio}
% Set default figure placement to htbp
\makeatletter
\def\fps@figure{htbp}
\makeatother
\setlength{\emergencystretch}{3em} % prevent overfull lines
\providecommand{\tightlist}{%
  \setlength{\itemsep}{0pt}\setlength{\parskip}{0pt}}
\setcounter{secnumdepth}{-\maxdimen} % remove section numbering
\ifLuaTeX
  \usepackage{selnolig}  % disable illegal ligatures
\fi

\title{Change in Head by Aquifer System}
\author{Annette Hilton}
\date{2023-06-05}

\begin{document}
\maketitle

\hypertarget{attach-packages}{%
\subsection{Attach Packages}\label{attach-packages}}

\begin{Shaded}
\begin{Highlighting}[]
\CommentTok{\# Attach packages }

\FunctionTok{library}\NormalTok{(tidyverse)}
\end{Highlighting}
\end{Shaded}

\begin{verbatim}
## Warning: package 'tidyverse' was built under R version 4.2.1
\end{verbatim}

\begin{verbatim}
## -- Attaching packages --------------------------------------- tidyverse 1.3.2 --
## v ggplot2 3.3.6     v purrr   0.3.4
## v tibble  3.1.7     v dplyr   1.0.9
## v tidyr   1.2.0     v stringr 1.4.0
## v readr   2.1.2     v forcats 0.5.1
## -- Conflicts ------------------------------------------ tidyverse_conflicts() --
## x dplyr::filter() masks stats::filter()
## x dplyr::lag()    masks stats::lag()
\end{verbatim}

\begin{Shaded}
\begin{Highlighting}[]
\FunctionTok{library}\NormalTok{(here)}
\end{Highlighting}
\end{Shaded}

\begin{verbatim}
## Warning: package 'here' was built under R version 4.2.1
\end{verbatim}

\begin{verbatim}
## here() starts at C:/Users/aeliz/Dropbox/Documents/github/chapter1_vhg
\end{verbatim}

\begin{Shaded}
\begin{Highlighting}[]
\FunctionTok{library}\NormalTok{(janitor)}
\end{Highlighting}
\end{Shaded}

\begin{verbatim}
## Warning: package 'janitor' was built under R version 4.2.1
\end{verbatim}

\begin{verbatim}
## 
## Attaching package: 'janitor'
## 
## The following objects are masked from 'package:stats':
## 
##     chisq.test, fisher.test
\end{verbatim}

\begin{Shaded}
\begin{Highlighting}[]
\FunctionTok{library}\NormalTok{(dplyr)}
\FunctionTok{library}\NormalTok{(knitr)}


\CommentTok{\# Disable scientific notation }

\FunctionTok{options}\NormalTok{(}\AttributeTok{scipen=}\DecValTok{999}\NormalTok{)}
\end{Highlighting}
\end{Shaded}

\hypertarget{read-in-data}{%
\section{Read in data}\label{read-in-data}}

\begin{itemize}
\tightlist
\item
  By each regional aquifer system that you can analyze based on criteria
  (n=25 samples)
\item
  Data that has already been sorted in aquifer unit, contains DEM
  information
\item
  Regional aquifer systems include:

  \begin{itemize}
  \tightlist
  \item
    Mississippi Embayment
  \item
    North Atlantic Coastal Plain
  \item
    Central Valley
  \item
    Dakota Aquifer System
  \item
    Floridan
  \item
    Houston-Gulf Coast
  \item
    Roswell
  \end{itemize}
\end{itemize}

\hypertarget{criteria-for-analysis}{%
\subsection{Criteria for analysis}\label{criteria-for-analysis}}

Regional systems will be examined by aquifer unit. Each aquifer unit
must have at least n=25 wells in both the pre-1910 and post-2010 time
periods

\hypertarget{mississippi-embayment}{%
\subsubsection{Mississippi Embayment}\label{mississippi-embayment}}

\begin{Shaded}
\begin{Highlighting}[]
\CommentTok{\# Pre{-}1910 clean file }

\NormalTok{early1900s\_me }\OtherTok{\textless{}{-}} \FunctionTok{read\_tsv}\NormalTok{(here}\SpecialCharTok{::}\FunctionTok{here}\NormalTok{(}\StringTok{"change\_head"}\NormalTok{, }\StringTok{"data"}\NormalTok{, }\StringTok{"1900s\_me.txt"}\NormalTok{))}
\end{Highlighting}
\end{Shaded}

\begin{verbatim}
## Rows: 579 Columns: 22
## -- Column specification --------------------------------------------------------
## Delimiter: "\t"
## chr  (4): id, keep, aquifer, status
## dbl (18): oid, ned10m_bilinear_cusa_albers102003, latitude, longitude, well_...
## 
## i Use `spec()` to retrieve the full column specification for this data.
## i Specify the column types or set `show_col_types = FALSE` to quiet this message.
\end{verbatim}

\begin{Shaded}
\begin{Highlighting}[]
\CommentTok{\# Modern 2010{-}2022 clean file }

\NormalTok{modern\_me }\OtherTok{\textless{}{-}} \FunctionTok{read\_tsv}\NormalTok{(here}\SpecialCharTok{::}\FunctionTok{here}\NormalTok{(}\StringTok{"change\_head"}\NormalTok{, }\StringTok{"data"}\NormalTok{, }\StringTok{"modern\_me.txt"}\NormalTok{))}
\end{Highlighting}
\end{Shaded}

\begin{verbatim}
## Rows: 4013 Columns: 29
## -- Column specification --------------------------------------------------------
## Delimiter: "\t"
## chr   (8): site_no, id_x, national_aquifer_x, local_aquifer_x, code, keep, a...
## dbl  (20): oid, dec_lat_va_x, dec_long_va_x, well_depth_va, median, level_ye...
## dttm  (1): level_date
## 
## i Use `spec()` to retrieve the full column specification for this data.
## i Specify the column types or set `show_col_types = FALSE` to quiet this message.
\end{verbatim}

\hypertarget{aquifer-units-to-analyze-in-me}{%
\subparagraph{Aquifer units to analyze in
ME:}\label{aquifer-units-to-analyze-in-me}}

\begin{itemize}
\tightlist
\item
  Middle Claiborne aquifer
\item
  Lower Claiborne confining unit
\end{itemize}

\hypertarget{middle-claiborne-aquifer}{%
\paragraph{Middle Claiborne Aquifer}\label{middle-claiborne-aquifer}}

Post-2010 data

\begin{Shaded}
\begin{Highlighting}[]
\CommentTok{\# Subtract median water level from topography to get the real water level }

\NormalTok{modern\_head\_me }\OtherTok{\textless{}{-}}\NormalTok{ modern\_me }\SpecialCharTok{\%\textgreater{}\%} 
  \FunctionTok{mutate}\NormalTok{(}\AttributeTok{head =}\NormalTok{ topo\_ft }\SpecialCharTok{{-}}\NormalTok{ median)}

\CommentTok{\# Keep only records in the mcaq, that are confined, and what we used in our regional analysis }

\NormalTok{mca\_mod }\OtherTok{\textless{}{-}}\NormalTok{ modern\_head\_me }\SpecialCharTok{\%\textgreater{}\%} 
  \FunctionTok{filter}\NormalTok{(aquifer }\SpecialCharTok{==} \StringTok{"mcaq"}\NormalTok{) }\SpecialCharTok{\%\textgreater{}\%} 
  \FunctionTok{filter}\NormalTok{(status }\SpecialCharTok{==} \StringTok{"confined"}\NormalTok{) }\SpecialCharTok{\%\textgreater{}\%} 
  \FunctionTok{filter}\NormalTok{(}\SpecialCharTok{!}\FunctionTok{grepl}\NormalTok{(}\StringTok{"Unconfined single aquifer"}\NormalTok{, code))}

\CommentTok{\# Final summaries }

\NormalTok{mca\_mod\_head }\OtherTok{\textless{}{-}}\NormalTok{ mca\_mod }\SpecialCharTok{\%\textgreater{}\%}  
  \FunctionTok{group\_by}\NormalTok{(aquifer) }\SpecialCharTok{\%\textgreater{}\%} 
\NormalTok{  dplyr}\SpecialCharTok{::}\FunctionTok{summarise}\NormalTok{(}
    \AttributeTok{min =} \FunctionTok{min}\NormalTok{(head), }
    \AttributeTok{max =} \FunctionTok{max}\NormalTok{(head), }
    \AttributeTok{quantile =} \FunctionTok{quantile}\NormalTok{(head),}
    \AttributeTok{mean =} \FunctionTok{mean}\NormalTok{(head), }
    \AttributeTok{median =} \FunctionTok{median}\NormalTok{(head),}
    \AttributeTok{standard\_deviation =} \FunctionTok{sd}\NormalTok{(head)}
\NormalTok{    )}
\end{Highlighting}
\end{Shaded}

\begin{verbatim}
## `summarise()` has grouped output by 'aquifer'. You can override using the
## `.groups` argument.
\end{verbatim}

\begin{longtable}[]{@{}
  >{\raggedright\arraybackslash}p{(\columnwidth - 12\tabcolsep) * \real{0.1081}}
  >{\raggedleft\arraybackslash}p{(\columnwidth - 12\tabcolsep) * \real{0.1216}}
  >{\raggedleft\arraybackslash}p{(\columnwidth - 12\tabcolsep) * \real{0.1216}}
  >{\raggedleft\arraybackslash}p{(\columnwidth - 12\tabcolsep) * \real{0.1486}}
  >{\raggedleft\arraybackslash}p{(\columnwidth - 12\tabcolsep) * \real{0.1216}}
  >{\raggedleft\arraybackslash}p{(\columnwidth - 12\tabcolsep) * \real{0.1216}}
  >{\raggedleft\arraybackslash}p{(\columnwidth - 12\tabcolsep) * \real{0.2568}}@{}}
\caption{Table 1. Mississippi Embayment, head of Middle Claiborne
aquifer summary statistics for Post-2010}\tabularnewline
\toprule
\begin{minipage}[b]{\linewidth}\raggedright
aquifer
\end{minipage} & \begin{minipage}[b]{\linewidth}\raggedleft
min
\end{minipage} & \begin{minipage}[b]{\linewidth}\raggedleft
max
\end{minipage} & \begin{minipage}[b]{\linewidth}\raggedleft
quantile
\end{minipage} & \begin{minipage}[b]{\linewidth}\raggedleft
mean
\end{minipage} & \begin{minipage}[b]{\linewidth}\raggedleft
median
\end{minipage} & \begin{minipage}[b]{\linewidth}\raggedleft
standard\_deviation
\end{minipage} \\
\midrule
\endfirsthead
\toprule
\begin{minipage}[b]{\linewidth}\raggedright
aquifer
\end{minipage} & \begin{minipage}[b]{\linewidth}\raggedleft
min
\end{minipage} & \begin{minipage}[b]{\linewidth}\raggedleft
max
\end{minipage} & \begin{minipage}[b]{\linewidth}\raggedleft
quantile
\end{minipage} & \begin{minipage}[b]{\linewidth}\raggedleft
mean
\end{minipage} & \begin{minipage}[b]{\linewidth}\raggedleft
median
\end{minipage} & \begin{minipage}[b]{\linewidth}\raggedleft
standard\_deviation
\end{minipage} \\
\midrule
\endhead
mcaq & -186.789 & 387.9793 & -186.78900 & 97.78385 & 83.14224 &
98.14278 \\
mcaq & -186.789 & 387.9793 & 35.18334 & 97.78385 & 83.14224 &
98.14278 \\
mcaq & -186.789 & 387.9793 & 83.14224 & 97.78385 & 83.14224 &
98.14278 \\
mcaq & -186.789 & 387.9793 & 172.20810 & 97.78385 & 83.14224 &
98.14278 \\
mcaq & -186.789 & 387.9793 & 387.97928 & 97.78385 & 83.14224 &
98.14278 \\
\bottomrule
\end{longtable}

\begin{Shaded}
\begin{Highlighting}[]
\CommentTok{\# Histogram }

\NormalTok{Head\_2010 }\OtherTok{\textless{}{-}}\NormalTok{ mca\_mod}\SpecialCharTok{$}\NormalTok{head}

\FunctionTok{hist}\NormalTok{(Head\_2010,}
     \AttributeTok{main=}\StringTok{"Hydraulic Head Post{-}2010"}\NormalTok{,}
     \AttributeTok{xlab=}\StringTok{"Hydraulic Head (ft)"}\NormalTok{,}
     \AttributeTok{xlim=}\FunctionTok{c}\NormalTok{(}\SpecialCharTok{{-}}\DecValTok{200}\NormalTok{, }\DecValTok{400}\NormalTok{))}
\end{Highlighting}
\end{Shaded}

\includegraphics{change_head_files/figure-latex/unnamed-chunk-5-1.pdf}

Pre-1910 data

\begin{Shaded}
\begin{Highlighting}[]
\CommentTok{\# The DEM of the artesian flowing wells will be used as minimum head }

\CommentTok{\# Keep only records in the mcaq, that are confined, artesian, and what we used in our regional analysis}

\NormalTok{mca\_1900s }\OtherTok{\textless{}{-}}\NormalTok{ early1900s\_me }\SpecialCharTok{\%\textgreater{}\%} 
  \FunctionTok{filter}\NormalTok{(aquifer }\SpecialCharTok{==} \StringTok{"mcaq"}\NormalTok{) }\SpecialCharTok{\%\textgreater{}\%} 
  \FunctionTok{filter}\NormalTok{(status }\SpecialCharTok{==} \StringTok{"confined"}\NormalTok{) }\SpecialCharTok{\%\textgreater{}\%} 
  \FunctionTok{filter}\NormalTok{(artesian }\SpecialCharTok{==} \DecValTok{1}\NormalTok{)}


\CommentTok{\# Final summaries }

\NormalTok{mca\_1900\_head }\OtherTok{\textless{}{-}}\NormalTok{ mca\_1900s }\SpecialCharTok{\%\textgreater{}\%}  
\NormalTok{  dplyr}\SpecialCharTok{::}\FunctionTok{summarise}\NormalTok{(}
    \AttributeTok{min =} \FunctionTok{min}\NormalTok{(topo\_ft), }
    \AttributeTok{max =} \FunctionTok{max}\NormalTok{(topo\_ft), }
    \AttributeTok{quantile =} \FunctionTok{quantile}\NormalTok{(topo\_ft),}
    \AttributeTok{mean =} \FunctionTok{mean}\NormalTok{(topo\_ft), }
    \AttributeTok{median =} \FunctionTok{median}\NormalTok{(topo\_ft),}
    \AttributeTok{standard\_deviation =} \FunctionTok{sd}\NormalTok{(topo\_ft)}
\NormalTok{    )}
\end{Highlighting}
\end{Shaded}

\begin{longtable}[]{@{}rrrrrr@{}}
\caption{Table 1. Mississippi Embayment, head of Middle Claiborne
aquifer summary statistics for Pre-1910}\tabularnewline
\toprule
min & max & quantile & mean & median & standard\_deviation \\
\midrule
\endfirsthead
\toprule
min & max & quantile & mean & median & standard\_deviation \\
\midrule
\endhead
63.04988 & 378.6241 & 63.04988 & 128.5041 & 110.6407 & 73.93757 \\
63.04988 & 378.6241 & 75.91588 & 128.5041 & 110.6407 & 73.93757 \\
63.04988 & 378.6241 & 110.64070 & 128.5041 & 110.6407 & 73.93757 \\
63.04988 & 378.6241 & 141.90125 & 128.5041 & 110.6407 & 73.93757 \\
63.04988 & 378.6241 & 378.62406 & 128.5041 & 110.6407 & 73.93757 \\
\bottomrule
\end{longtable}

\begin{Shaded}
\begin{Highlighting}[]
\CommentTok{\# Histogram }

\NormalTok{Head\_1900s }\OtherTok{\textless{}{-}}\NormalTok{ mca\_1900s}\SpecialCharTok{$}\NormalTok{topo\_ft}

\FunctionTok{hist}\NormalTok{(Head\_1900s, }
     \AttributeTok{main=}\StringTok{"Hydraulic Head Pre{-}1910 (Topography Proxy)"}\NormalTok{,}
     \AttributeTok{xlab=}\StringTok{"Hydraulic Head (ft)"}\NormalTok{,}
     \AttributeTok{xlim=}\FunctionTok{c}\NormalTok{(}\SpecialCharTok{{-}}\DecValTok{200}\NormalTok{, }\DecValTok{400}\NormalTok{))}
\end{Highlighting}
\end{Shaded}

\includegraphics{change_head_files/figure-latex/unnamed-chunk-8-1.pdf}

\end{document}
